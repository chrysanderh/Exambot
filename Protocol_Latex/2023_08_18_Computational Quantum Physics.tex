\documentclass{article}%
\usepackage[T1]{fontenc}%
\usepackage[utf8]{inputenc}%
\usepackage{lmodern}%
\usepackage{textcomp}%
\usepackage{lastpage}%
%
\usepackage[svgnames]{xcolor}%
\usepackage{enumitem}%
\usepackage{ragged2e}%
\usepackage[breakable]{tcolorbox}%
\usepackage{graphicx}%
\usepackage[left=1in, right=1in, top=1in, bottom=1in]{geometry}%
\usepackage{eso-pic}%
\usepackage{tikz}%
\newcommand\Watermark{\put(0, 0.5\paperheight){\parbox[c][\paperheight]{\paperwidth}{\centering \rotatebox[origin=c]{45}{\tikz\node[opacity=0.1]{\includegraphics[width=0.7\textwidth]{2023_04_QEC.png}};}}}}%
\newtcolorbox{mycolorbox}{breakable, colback=Gainsboro, coltext=black, opacityfill=0.15, boxsep=0pt, arc=0pt, boxrule=0pt, left=0pt, right=0pt, top=2pt, bottom=2pt, nobeforeafter, fontupper=\normalsize, before upper={\begin{justify}\parindent0pt}, after upper={\end{justify}},}%
\AddToShipoutPictureBG{\Watermark}%
%
\begin{document}%
\normalsize%
\begin{center}%
\begin{Large}%
\textbf{Computational Quantum Physics}%
\end{Large}%
\linebreak%
\end{center}%
\begin{center}%
\begin{large}%
\textbf{Spring 2021, Examiner: M. H. Fischer}%
\end{large}%
\end{center}%
\begin{enumerate}%
\item%
\begin{mycolorbox}%
\textbf{Summary:}%
\newline%
Just had my exam:\newline%
Numerov, short derivation, scattering problem, bound states\newline%
Single particle time evolution (unitary, spectral, split{-}operator (complexity of FFT))\newline%
larger problem: trotterization for XXZ model\newline%
even larger: TEBD: area law, algorithm, error sources%
\end{mycolorbox}%
\end{enumerate}%
\newpage%
\begin{center}%
\begin{large}%
\textbf{Spring 2020, Examiner: T. Neupert, M. H. Fischer}%
\end{large}%
\end{center}%
\begin{enumerate}%
\item%
\begin{mycolorbox}%
\textbf{Summary:}%
\newline%
Hi guys, I just had the cqp exam.\newline%
Overall I would say the exam was quite nice. The setting is a classroom, you have a chalk board and both teachers are sitting in the back and are asking questions. Incomplete list of topics:\newline%
Time evolution (unitary approximation)\newline%
Split operator\newline%
MPS\newline%
DMRG\newline%
TEBD\newline%
Multiplying two spins to get a 4x4 Hamiltonian\newline%
Spin Monte Carlo and cluster updates\newline%
The exam was over before I knew it.\newline%
About reaching the location: I took the tram to the "Universität Irchel" Stop, From there it's about 7 minutes walking to the building. Go immediately left and down when entering the campus, the building should be indicated with a sign (YI 1), then it's up one stair to the correct floor.%
\end{mycolorbox}%
\item%
\textbf{Summary:}%
\newline%
They asked me (about) the following topics/things:\newline%
Can you write down the Heisenberg Hamiltonian for two spins?\newline%
Can you write it in a matrix form?\newline%
What does it mean for the matrix to be block diagonal?\newline%
What would you do in case we have a spin chain of like 20, 30, 40 spins? ({-}> Lanczos)\newline%
Can you explain what Lanczos algorithm does and how it works?\newline%
If we consider even larger systems with 40+ spins, what would be the next preferred method to find the ground state?\newline%
Can you explain how to construct an MPS?\newline%
I was very surprised that in my exam these were the only topics covered (no DMRG, TEBD, (Q)MC, DFT, etc.), but I feel this differs a lot between the exams.\newline%
Especially Titus is very encouraging during the exam by the way, which I found very nice ;).%
\end{enumerate}%
\newpage%
\end{document}