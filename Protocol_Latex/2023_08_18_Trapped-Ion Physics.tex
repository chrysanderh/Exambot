\documentclass{article}%
\usepackage[T1]{fontenc}%
\usepackage[utf8]{inputenc}%
\usepackage{lmodern}%
\usepackage{textcomp}%
\usepackage{lastpage}%
%
\usepackage[svgnames]{xcolor}%
\usepackage{enumitem}%
\usepackage{ragged2e}%
\usepackage[breakable]{tcolorbox}%
\usepackage{graphicx}%
\usepackage[left=1in, right=1in, top=1in, bottom=1in]{geometry}%
\usepackage{eso-pic}%
\usepackage{tikz}%
\newcommand\Watermark{\put(0, 0.5\paperheight){\parbox[c][\paperheight]{\paperwidth}{\centering \rotatebox[origin=c]{45}{\tikz\node[opacity=0.1]{\includegraphics[width=0.7\textwidth]{2023_04_QEC.png}};}}}}%
\newtcolorbox{mycolorbox}{breakable, colback=Gainsboro, coltext=black, opacityfill=0.15, boxsep=0pt, arc=0pt, boxrule=0pt, left=0pt, right=0pt, top=2pt, bottom=2pt, nobeforeafter, fontupper=\normalsize, before upper={\begin{justify}\parindent0pt}, after upper={\end{justify}},}%
\AddToShipoutPictureBG{\Watermark}%
%
\begin{document}%
\normalsize%
\begin{center}%
\begin{Large}%
\textbf{Trapped{-}Ion Physics}%
\end{Large}%
\linebreak%
\end{center}%
\begin{center}%
\begin{large}%
\textbf{Spring 2023, Examiner: Daniel Kienzler}%
\end{large}%
\end{center}%
\begin{enumerate}%
\item%
\begin{mycolorbox}%
\textbf{Summary:}%
\newline%
\#\#\# Ion Trapping\newline%
1. Can we confine a particle with a static electric field?\newline%
2. What can we do about it?\newline%
3. Explain how the Paul trap works.\newline%
4. When does the pseudopotential approximation hold?\newline%
5. Explain stability with the appropriate diagram.\newline%
6. What textbook system can we use do describe our trapped particle? (QHO)\newline%
7. How does it look like for 2 ions? (normal modes)\newline%
8. Describe decoherence mechanisms for the system (dephasing, heating)\newline%
\newline%
\#\#\# Spin motion coupling\newline%
9. Let's talk about spin motion coupling. (free to lead from here, I started from the trapped ion interaction Hamiltonian in the z direction and did the Lamb Dicke Expansion)\newline%
10. In the atom field interaction, what is the field? (laser controlled by us)\newline%
11. What is %
$z_{0}$%
? (ground state wave packet size)\newline%
12. Why can we expand %
$e^{ik_{z}Z}$%
? (%
$\eta$ %
 small, which we can see from typical laser wavelengths and ground state wave packet sizes)\newline%
13. What does this give to us now? (sidebands)\newline%
14. We have all the Hamiltonians now, but none of them include frequency. How do we get the sideband frequencies? (switch to interaction picture)\newline%
\newline%
\#\#\# Laser cooling\newline%
15. How does cooling with the red sideband work?\newline%
16. Can we just use the carrier and red sideband to drive the cooling process? Why do we need a dissipative (non{-}unitary) process?\newline%
\newline%
\#\#\# State dependent forces\newline%
17. How can we use the sidebands to implement state dependent forces?\newline%
18. Where does the state dependency come from?%
\newline%
\newline%
\textbf{Exam atmosphere:}%
\newline%
The atmosphere was relaxed, the examiner sometimes asked specific questions, sometimes also just gave a broad topic you can jump off from.%
\end{mycolorbox}%
\end{enumerate}%
\newpage%
\begin{center}%
\begin{large}%
\textbf{Spring 2021, Examiner: Daniel Kienzler}%
\end{large}%
\end{center}%
\begin{enumerate}%
\item%
\begin{mycolorbox}%
\textbf{Summary:}%
\newline%
Hey! here my exam:\newline%
 Trapping:\newline%
 {-}why can't we trap in 3D with only static potentials\newline%
 {-}what we can do about this\newline%
 {-}effect of noisy electric field\newline%
 {-}trap stability\newline%
 {-}motion of multiple ions in the same trap + noise for com mode and stretch mode\newline%
 Laser cooling:\newline%
 {-}doppler for going into LD regime\newline%
 {-}sideband cooling\newline%
 {-}why we need the dissipation and we can't use carrier\newline%
 • QCCD:\newline%
 {-}what it is, what are the challenges, what problems of the long string approach it fixes e State dependent forces in X{-}basis:\newline%
 {-}how are they implemented\newline%
 {-}how can you use these for performing and entangling gate\newline%
 {-}one loop vs two loops in ms gate\newline%
 • Quantum logic spettroscopy:\newline%
 {-}protocol for studying H2+ (very very briefly)%
\end{mycolorbox}%
\item%
\textbf{Summary:}%
\newline%
Hi! Just had my exam and we talked about the following:\newline%
 Talk about trapping, specifically Paul trap\newline%
 why can't we use static potential? (Laplace)\newline%
 How does Paul trap overcome this problem?\newline%
 How does trapping for 1 atom look like? Simple harmonic oscillator\newline%
 How many oscillators do we have when trapping one ion?\newline%
 What happens when we have two ions in the trap? How do they couple?\newline%
 How does decoherence affect the ions? (heating and decoherence)\newline%
 Explain heating further. (electrodes cause fluctuating E field)\newline%
 How would that affect the 2 ions?\newline%
 Interaction btw spin and motion (talked about Ch 3, atom field interaction, dipole approximation, Lamb{-}Dicke expansion, ...)\newline%
 What does the LD parameter mean?\newline%
 What limit are we looking at?\newline%
 What happens to the Hamiltonian in this limit? Talked about carrier transition, red, blue, ... (only wanted me to write down the red side band H)\newline%
 How can the RSB H be used for cooling?\newline%
 Why can't we use the carrier and RSB H only?\newline%
 Why non{-}unitary?\newline%
 Talked about state{-}dependent forces (x dependent)%
\item%
\begin{mycolorbox}%
\textbf{Summary:}%
\newline%
Had my exam today, went as follows:\newline%
 I) Start with trapping\newline%
 {-}Why can't we just a static potential in 3D? (Laplace)\newline%
 {-}HOW can we create confinement then?\newline%
 {-}How does the RF potential create confinement?\newline%
 2) On to stability in trapping\newline%
 {-}Sketch the stability diagram (1.7). Started sketching it but he wasn't really that keen on drawing it in details, rather just the following questions:\newline%
 {-}When do we have stable trapping?\newline%
 {-}What are a and q params (axes)? What physical parameters are they related to?\newline%
 {-}What boundaries do we have for the trapping? (Beta {-}> O or 1) and what is the physical meaning of that?\newline%
 3) Had a brief talk about noise in trapping (noisy electric field).\newline%
 {-}What two decoherence mechanisms that leads to (heating, dephasing). Gave the Lindblad operators for that.\newline%
 5) Then on to how we can control the motion and the spin.\newline%
 {-} Didn't quite not what he was fishing for. Mentioned carrier transition and sidebands so he went for that.\newline%
 {-}He asked about what regime we are in? (LD) What does that mean, etc?\newline%
 6) Then finally on to sideband cooling\newline%
 {-}how can the red sideband be used to cool?\newline%
 {-}Why do we need to scatter a photon?\newline%
 {-}Why does it have to be non{-}unitary?\newline%
 {-}Why can't we use a RSB and the carrier? How would that look for a thermal state? (ended up sketching it on a state ladder)\newline%
 {-}Why do other transitions not scatter a photon, i.e. why only (predominantly) the carrier? (decay along sideband transition suppressed by eta\^{}2)\newline%
 He didn't show me any graphs or figures, just asked questions and maybe drew something on the blackboard.%
\end{mycolorbox}%
\end{enumerate}%
\newpage%
\end{document}