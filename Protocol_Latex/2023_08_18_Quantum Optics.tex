\documentclass{article}%
\usepackage[T1]{fontenc}%
\usepackage[utf8]{inputenc}%
\usepackage{lmodern}%
\usepackage{textcomp}%
\usepackage{lastpage}%
%
\usepackage[svgnames]{xcolor}%
\usepackage{enumitem}%
\usepackage{ragged2e}%
\usepackage[breakable]{tcolorbox}%
\usepackage{graphicx}%
\usepackage[left=1in, right=1in, top=1in, bottom=1in]{geometry}%
\usepackage{eso-pic}%
\usepackage{tikz}%
\newcommand\Watermark{\put(0, 0.5\paperheight){\parbox[c][\paperheight]{\paperwidth}{\centering \rotatebox[origin=c]{45}{\tikz\node[opacity=0.1]{\includegraphics[width=0.7\textwidth]{2023_04_QEC.png}};}}}}%
\newtcolorbox{mycolorbox}{breakable, colback=Gainsboro, coltext=black, opacityfill=0.15, boxsep=0pt, arc=0pt, boxrule=0pt, left=0pt, right=0pt, top=2pt, bottom=2pt, nobeforeafter, fontupper=\normalsize, before upper={\begin{justify}\parindent0pt}, after upper={\end{justify}},}%
\AddToShipoutPictureBG{\Watermark}%
%
\begin{document}%
\normalsize%
\begin{center}%
\begin{Large}%
\textbf{Quantum Optics}%
\end{Large}%
\linebreak%
\end{center}%
\begin{center}%
\begin{large}%
\textbf{Fall 2022, Examiner: Atac Imamoglu}%
\end{large}%
\end{center}%
\begin{enumerate}%
\item%
\begin{mycolorbox}%
\textbf{Summary:}%
\newline%
Atac started with asking me different ways to couple a laser to the center of mass motion and collective motion of matter and I talked a bit about cavity optomechanics and here he asked if the mirror should be heavy or not have a large coupling rate. Then he moved on to trapped ions, because this was what he wanted hear, (he said something that using cavity optomechanics to cool down isn't really used), and here I struggled a bit more. Then he moved on~to~the~exercise where I had this master equation: %
$d\rho/dt = continous-wave drive + K_c/2 L{\rho, a_C}  + \Gamma_1 / 2 L{\rho, a_c^dagger, a_c} + \Gamma_2 /2 L{\rho, a_c^2}$%
. This was describing a black{-}box inside a cavity, and light coming out of the cavity went into an HBT experiment. He wanted me to explain what each term represents and how this affects the results in the HBT experiment, I had no clue what would be the effect of Gamma1 and he didn't give me the answer so this remains~a~mystery~to~me%
\newline%
\newline%
\textbf{Exam atmosphere:}%
\newline%
I started well, but then things went south, and he didn't try to help me, at some point we were staring at each other, not giving me any clue on how to move one. He doesn't give any hint if you are giving the correct answer or not, so at the end I didn't know how well or bad the exam went.%
\end{mycolorbox}%
\end{enumerate}%
\newpage%
\begin{center}%
\begin{large}%
\textbf{Fall 2020, Examiner: Jonathan Home}%
\end{large}%
\end{center}%
\begin{enumerate}%
\item%
\begin{mycolorbox}%
\textbf{Summary:}%
\newline%
just had my exam.\newline%
Choose a topic: Master equation, explain the derivation until the three approximation. Then the professor showed me a picture of a master equation in Iinblad form, he asked what is the expression of Iinblad operator in this case. Then he moved on to a picture of spectrum of g2 function, asked about g2 expression and explanation. Then a spectrum of rabi oscillation.%
\end{mycolorbox}%
\item%
\textbf{Summary:}%
\newline%
Topic to talk about (JC Hamiltonian): Approximations, eigenstates, eigenenergies (with detuning O).\newline%
 • g2 correlation function: expression and graph\newline%
 • Spectrum: Mollow triplet + dressed states pictures\newline%
 •Coherent states: statistics + collapse and revival\newline%
 •Sideband cooling graph\newline%
 •Lamb{-}dicke parameter\newline%
 •Cavity optomechanics: why do we use small mirrors held by wires.%
\item%
\begin{mycolorbox}%
\textbf{Summary:}%
\newline%
I started with the JC hamiltonian, eigenstates/ energies at resonance, the approximations: RWA as well as the ones for the elect. dipole hamiltonian, and why there is only one mode he showed a master equ.: Lindblad op. and some evaluation of mean values figure with coherent/squeezed coherent states: how the distribution looks like figure of u\_ms, v\_ms and figure detuning vs temperature: scattering force, doppler cooling, cooling rate, why not zero temperature, heating, random walk%
\end{mycolorbox}%
\item%
\textbf{Summary:}%
\newline%
{-} start with topic you want, JC Hamiltonia, pretty much the same as above, but he talked to me alot about the g factor, then energy and eigenstates in resonance\newline%
{-} cat states, what happens when xou apply a imaginary displacement to a cat stat?\newline%
{-} squeezed and coherent state distribution \& squeezed coherent states%
\item%
\begin{mycolorbox}%
\textbf{Summary:}%
\newline%
{-} Sideband cooling (hamiltonian derivation, lamb dicke parameter approx)\newline%
{-} Cavity Optomechanics analogy (explain setup with a mirror suspended by some small cables)\newline%
{-} Coherent and Squeezed number distribution.\newline%
{-} Collapse and revival comparison for coherent vs squeezed (how does tc, tr and amplitude of revivals change)%
\end{mycolorbox}%
\item%
\textbf{Summary:}%
\newline%
jaynes cummings: what approximations we made in dipole approximations, what in JC, eigenstates and eigenvalues, coupling \newline%
Coherent vs. Squeezed states, plotred in Q function, collapse and revival of rabi oscillayions and which are faster\newline%
Doppler cooling: graph with temperature vs detuning, u,v,w plotted as a function of detuning\newline%
Master equation for cavity decay (was shown on a slide, i had to comment)\newline%
g2 correlation function: g2(O), rabi oscillations, envelope (decay)%
\item%
\begin{mycolorbox}%
\textbf{Summary:}%
\newline%
{-} First I started explaining the topic of choice, single photon pulses in my case.\newline%
{-} Then he asked about the Hong{-}Ou{-}Mandel effect and the Purcell effect (things I mentioned when explaining single photon pulses)\newline%
{-} I was asked to write a complete Hamiltonian of an atom in a cavity that couples to the environment and then forgetting the contribution of the coupling to the environment he asked about RWA and dressed states\newline%
{-} Then we moved to the hamiltonian in the dispersive limit and for the evolution of a coherent state in this regime.\newline%
{-} We spoke about cats and the state of the system at different stages in the Haroche experiment.\newline%
{-} He showed the plot of the population in excited state of an atom in a trap as a function of the laser freq with the missing peak in the RSB.\newline%
{-} Finally, in the remaining 3 min, he showed the g(2) plot and asked me to comment on it.\newline%
I completely agree with the fact that Jonathan is a very nice examiner that barely interrupts and helps when needed. Very good luck!%
\end{mycolorbox}%
\item%
\textbf{Summary:}%
\newline%
He asked what I wanted to start with, and I went for the Purcell effect. starting with the setup description, I wrote the interaction Hamiltonian and he asked where it comes from\newline%
had to go back to dipole approximation and how we obtain the JC form (complete hamiltonian, interaction pictures, RWA + conditions)\newline%
he asked what approximations we did to get the dipole approx (l comented the neutral atom and the long wavelength approx)\newline%
then he asked for me to explain the kappa\newline%
I had to talk about the coupling of the cavity with multiple modes (l first wrote the interaction for a TLS with many modes, and had to correct the sigmas for a operators)\newline%
once this done I could go back to Purcell:\newline%
Liouville equation + master equation\newline%
I started writing the results of the OBE bu he didn't want all, just the result\newline%
I wrote the resulting exponential decay of rho\_ll\newline%
Described the comparison to free{-}field decay\newline%
He asked what it is good for, I comented single photon pulses and HOM, he wanted me to describe the HOM 5050{-}BS + formula of a single photon pulse, he didn't want the whole derivation, so just the idea of bunching\newline%
Then he started with graphs on his ipad:\newline%
Q functions of coherent state + 2 squeezed states: what are the number distributions, and how does squeezing affect them a probability distribution of a fock state in x: what features I see that would tell me something about the wigner function (there he helped, first he asked me to write what the Q function would be for a fock state, but really wanted me to say that the probability distributions can be obtained as marginals of the wigner function,and the negativity due to the nodes in the wavefunction)\newline%
g(2) graph and wanted me to explain the different features\newline%
he asked me to draw and explain the mollow triplet + the conceptual explanation of the 3 peaks (using the dressed state picture) + aksed about the splitting between the + and {-}, and why it is "constant" (laser {-}> large coherent state {-}> narrow distribution)%
\item%
\begin{mycolorbox}%
\textbf{Summary:}%
\newline%
{-} I started with the master equations and just explained how we derived them and he wanted to know the three approximations. Then I had to write the final form and the initial interaction hamiltonian.\newline%
{-} asking what approximation we did to get to such a hamiltonian (Jaynes{-}Cummings, RWA) and which approximation and transformation we did to get to the electric dipol hamiltonian\newline%
{-} he showed me the g(2) function as well and i had to interpret it and afterwards explain the spectrum of such an emitter (resonant fluorescent)\newline%
{-} we went on to coherent states, their energy distribution and their number distribution. Then he showed some squeezed states which were displaced with the same alpha and ask about the number distribution of them.\newline%
{-} in the end I also had to talk about doppler cooling and he showed me also the graph with the temperature of the atom and the graph of u\_m, v\_m. He wanted me to explain the different regimes, when is dipol force dominant and when is the scattering force dominant.%
\end{mycolorbox}%
\item%
\textbf{Summary:}%
\newline%
Hi everybody, I just had my QO exam. Jonathan is a really nice examiner in my opinion and happily helps you out when you're stuck. Here's a description of the questions asked:\newline%
 • His first question was if there is any specific topic I'd like to talk about. So if you didn't know that yet, prepare a short "presentation" on a topic.\newline%
 • I chose the Jaynes{-}Cummings model and questions included: What is g and can you express it in constants? How can you make the coupling G) stronger? (Decrease the volume{]} What does the RWA exactly entail? What variable has to be smaller than what other variable for it to hold? What are the eigenstates? What do the dressed states look like when you apply a coherent field? Scattering off an atom in a field. Can you illustrate this? (Mollow triplet) How far are the peaks apart? What happens in the weak/strong driving regime? What do the dressed states look like in the case of scattering (strong driving regime)?\newline%
 • Jonathan showed a picture of the second order correlation function and asked if I could comment on it (zero fortau O, Rabi oscillations, converges to 1, writing down the normalized formula. explaining why the ordering of operators in this formula is important, etc,)\newline%
 • Jonathan showed a picture of a plot of a squeezed state using the Q{-}function and asked me to comment on it. How can you detect it? What would you measure?%
\item%
\begin{mycolorbox}%
\textbf{Summary:}%
\newline%
Hi everybody. I just had the exam. I can just repeat that Jonathan was really nice and helps out. Here is a description of the exam:\newline%
He asked me about a topic I'd like to talk about, I chose forces due to interaction with light\newline%
He had two graphs ready faster than I could blink. The graphs showed the plot of the u\_m. and which we had in the lecture and another which showed the temperature plotted against detuning and he wanted me to talk a little bit about them\newline%
He then showed me the spectrum of the sideband cooling with the missing peak for the red sideband, which I should explain and I think he then wanted me to explain how the spectrum would look like for a Doppler cooled system (different binding regimes)\newline%
We talked about the {[}am{-}dicke parameter. size/ approximation..\newline%
He then showed me a plot of a Q•function of a squeezed state (l think the same question as for Sam)\newline%
Then he showed a plot of g(2), which I should explain and talk about the spectrum of such a single emitter\newline%
Hope that helps. good luck!%
\end{mycolorbox}%
\item%
\textbf{Summary:}%
\newline%
Hi all. my exam topics/questions:\newline%
Jaynes{-}Cummings Hamiltonian\newline%
Electric dipole Hamiltonian\newline%
Goeppert{-}Meyer + Long wavelength\newline%
Rotating wave approximation\newline%
Factors in %
$g$ %
 and how to enhance it\newline%
Eigenstates Of JC at resonance and off{-}resonance\newline%
Home showed a graph Of spectrum Of a three level system, i.e. one transition was probed one was driven, strongly driven at resonance and asked What transition was driven\newline%
He showed a similar graph off resonance and asked me to identify the peaks corresponding to excited/ground states (decay excited state %
$\Rightarrow$ %
 broader peak)\newline%
Autocorrelation function and its normalization\newline%
Normal ordering, why (rotating wave)\newline%
He showed a graph of and asked me to comment on it\newline%
Power broadening%
\item%
\begin{mycolorbox}%
\textbf{Summary:}%
\newline%
Hi everybody, here are the questions I was asked:\newline%
As a first topic I chose the Master equation. He asked me to explain the assumptions and part of the derivation. He also wanted me to explain how we obtained the Jaynes{-}Cummings Hamiltonian,\newline%
Then he showed me another Master equation (l think for the coupling of a cavity to the environment) and asked if I could write the Lindblad operators and how I would obtain the mean occupancy from this equation.\newline%
He showed a picture of a squeezed state and wanted to know how I would measure it (balanced homodyne experiment setup„..)\newline%
The next question was about the number distributions of a coherent state and squeezed states.\newline%
He showed me a graph with Rabi oscillations and wanted me to explain it: collapse, revival (also for squeezed states)%
\end{mycolorbox}%
\item%
\textbf{Summary:}%
\newline%
JC Hamiltonian. Electric dipole H. g constant. eigenstates at resonance and in dispersive regime\newline%
Home showed a graph of spectrum of a three level system. i.e. one transition was probed one was driven. strongly driven at resonance and asked what transition was driven\newline%
Autocorrelation function, normal ordering\newline%
Showed a graph of and asked me to comment on it\newline%
Corresponding spectrum, mollow triplet, dressed states\newline%
Squeezed state, how to measure it%
\item%
\begin{mycolorbox}%
\textbf{Summary:}%
\newline%
jaynes cummings: what approximations we made in dipole apptoximations, what in JC, eigenstates and eigenvalues, coupling\newline%
Coherent vs. Squeezed states, plotred in Q function, collapse and revival of rabi oscillayions and which are faster\newline%
Doppler cooling: graph With temperature vs detuning, plotted as a function Of detuning\newline%
Master equation for cavity decay (was shown on a slide, i had to comment)\newline%
g2 correlation function: g2(O), rabi oscillations, envelope (decay)%
\end{mycolorbox}%
\item%
\textbf{Summary:}%
\newline%
My topics were:\newline%
 {-}Jaynes{-}Cummings, write down Hamiltonian, name all the approximations made, write down eigenenergies \& eigenstates, what would these look like for detuning, dispersive limit\newline%
 {-}Collapse \& revival, coherent \& squeezed states (horizontally and vertically), how would collapse \& revival behave with those initial states\newline%
 {-}Got shown graphs (one with u \& v, the other with temp vs detuning) and had to comment on them, explain Doppler cooling in the process\newline%
 {-}How sideband cooling works (topic introduced via the graph where there is no peak for red{-}detuned laser), Hamiltonians, Lamb{-}Dicke parameter \& limit, comparison to optomechanical system%
\item%
\begin{mycolorbox}%
\textbf{Summary:}%
\newline%
First he asked me to choose a topic to Start with, I chose the Jaynes{-}Cummings, he laughed as everyone chooses that, but we continued nevertheless.\newline%
{-} I needed to write the Hamiltonian, explain why we can use only one field mode operator, explain the RWA, write the eigenstates and the eigenenergies.\newline%
{-} Dispersive regime, what are the eigenstates, how can this hamiltonian be viewed as (as the change in frequency of the light and/or as the change in the phase of the atom)\newline%
{-} Then he showed me a picture of the lambda system and a graph very similar to the Ro\_ge distribution in problem set 5 and asked me to explain the peaks in the graphs and which eigenstate predominates in the two peaks in the graph and when do they occour (Raman resonance condition and EIT basically)\newline%
{-} We moved on to the 2nd order correlation function {-} how is it defined, its properties for a single quantum emitter, two emmiters and large number of emmiters, why is it O at TZO for a single emitter. He also showed me a graph of one such 2nd order correlation function and asked me what produces the wiggles (Rabi oscillations) and what produces the envelope (decay), and also what would the power spectrum look like and why (mollow triplet)\newline%
{-}Then he showed me a graph from 12th slide of lecture 23 (laser cooling in the lamb{-}dicke limit) and asked me to explain why is it like that (that there is no transition in the red sideband) and to write the hamiltonian in the red sideband. He also then asked me what regime we are working in here and what is the difference in conditions of this regime and the one with the dipole and the scattering force.\newline%
{-} What do we get when we combine the red sideband (JC) Hamiltonian and the blue sideband (aJC) Hamiltonian {-} State dependent forces.\newline%
{-}What can we do with this Hamiltonian (create cat states of the mechanical modes of the trapped particle) and how (show the whole proces and doing the measurement in the e/g basis to get the cat states)\newline%
{-} What happens if we don't do the measurement, how does the density matrix look like and what does its Q{-}function look like then (two gaussians)%
\end{mycolorbox}%
\item%
\textbf{Summary:}%
\newline%
Home asked me if I wanted to start with a topic and I briefly described the HOM effect, the experiment schematics, the unitary transformation and the derivation of the result. He interupted to ask me which modes I'm summing over when I wrote the expression for the wavepacket. I explained that we considered broadband transformation, though i knew he wanted to hear something else. Then he aked why I don't have vectors in my expression, which resulted in me confusingly writing arrows on my k. Anyways, he told me he was referring to considering only logitudinal modes (if I remember correctly). I also draw the HOM dip.\newline%
Then he asked me what will happen if we had 2 photons in one of the ports and I went on to speak about the second order correlation function and the photon antibunching effect. He asked me about the timescale of the oscillations and why they decohere. I went back to explain that for 2 emitters we don't have g(2) = O. Then Home asked me to write an expression for the second order correlation function (which I did) and his follow up was "Why do we have this ordering ot operators in particular?" (normal ordering and how photon detectors work) I wasn't too convincing but he looked rather happy :D\newline%
Then he showed me fig. 10.1 and another graph of temperature and detuning which we haven't seen. I briefly explained doppler cooling. then he asked me about heating and explained the spontaneous emission and the random walk, however he was prompting towards broader distribution of velocities (I think). Lastly he showed me collapse and revival of Rabi oscillations graphs and asked me to comment on those (superposition of fock states, decoherence, mean values). I also wrote the expression for a coherent state as a superpostion of fock states.\newline%
There were 3 more minutes left and he was searching for another thing to show me but he decided not to and that it was enough. Not sure if this is a good or a bad thing :D\newline%
All in all, Home is a very nice examiner and he doesn't go into too much details and let's you speak and really show what you know. (I'll see if my expression was correct after I got my grade though)%
\end{enumerate}%
\newpage%
\begin{center}%
\begin{large}%
\textbf{Fall 2019, Examiner: Tilman Esslinger}%
\end{large}%
\end{center}%
\begin{enumerate}%
\item%
\begin{mycolorbox}%
\textbf{Summary:}%
\newline%
I just took the exam directed by Prof. Esslinger. The atmosphere was quite good and he was nice. He first asked me to describe light matter interaction. I wrote down the classical Hamiltonian and then quantized it. It took a long time and then he directly asked me to describe optical Bloch equation. Then he asked when the phase of the dipole would play a role (in a atom clock) and what cause the decay (the spontaneous emission). After that he asked me to describe the evolution of Bloch vector on the sphere (rotating and decaying).\newline%
He asked me to describe the photon echo and asked what may cause the sudden disappearance of the echo when heating up the sample and the mechanism of it (phase transition and energy splitting of the level). Finally he asked me about atom interferometer and asked what would happen if a spontaneous light is placed on the path of the atom beam.%
\end{mycolorbox}%
\item%
\textbf{Summary:}%
\newline%
This morning I had Quantum Optics, here is how it went:\newline%
Asks first how matter interacts with light, dynamics, Bloch equations, etc. Then how to do a two{-}level system, when the RWA breaks down, asks to draw spectrum of alkali atom, why we have to do the splitting, selection rules, why they are important, overlap between different levels (that are not selection{-}rule forbidden), then shift to free subject (l choose Rydberg atoms, blockage, paper by Grangier group, speak about it), then he asks what happens when you add a third atom, I sketch out a Hamiltonian, Esslinger goes like "honestly I don't know, there's a paper coming out on it soon", then squeezing, Heisenberg uncertainty, non{-}linear medium, homodyne detection. You can really steer the exam where you want it to go, he will not really stop you from talking unless he wants a precision on something. At first, I was a little nervous (my pen was shaking haha) but it went gradually away with Esslinger laughing and creating a good mood.%
\end{enumerate}%
\newpage%
\end{document}