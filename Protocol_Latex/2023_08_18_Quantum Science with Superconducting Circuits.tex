\documentclass{article}%
\usepackage[T1]{fontenc}%
\usepackage[utf8]{inputenc}%
\usepackage{lmodern}%
\usepackage{textcomp}%
\usepackage{lastpage}%
%
\usepackage[svgnames]{xcolor}%
\usepackage{enumitem}%
\usepackage{ragged2e}%
\usepackage[breakable]{tcolorbox}%
\usepackage{graphicx}%
\usepackage[left=1in, right=1in, top=1in, bottom=1in]{geometry}%
\usepackage{eso-pic}%
\usepackage{tikz}%
\newcommand\Watermark{\put(0, 0.5\paperheight){\parbox[c][\paperheight]{\paperwidth}{\centering \rotatebox[origin=c]{45}{\tikz\node[opacity=0.1]{\includegraphics[width=0.7\textwidth]{2023_04_QEC.png}};}}}}%
\newtcolorbox{mycolorbox}{breakable, colback=Gainsboro, coltext=black, opacityfill=0.15, boxsep=0pt, arc=0pt, boxrule=0pt, left=0pt, right=0pt, top=2pt, bottom=2pt, nobeforeafter, fontupper=\normalsize, before upper={\begin{justify}\parindent0pt}, after upper={\end{justify}},}%
\AddToShipoutPictureBG{\Watermark}%
%
\begin{document}%
\normalsize%
\begin{center}%
\begin{Large}%
\textbf{Quantum Science with Superconducting Circuits}%
\end{Large}%
\linebreak%
\end{center}%
\begin{center}%
\begin{large}%
\textbf{Fall 2020, Examiner: Christopher Eichler}%
\end{large}%
\end{center}%
\begin{enumerate}%
\item%
\begin{mycolorbox}%
\textbf{Summary:}%
\newline%
I just had QSSC. He started with the 5 criteria, then he asked me about the different ways to initialize, then about the readout mechanism (also the formulae for g and kappa) \newline%
How to fabricate a josephson junction (sketch the different layers like on the slide)\newline%
C phase gate, how to implement it\newline%
Distance 2 surface code, explain how a 4 qubit X operator works.%
\end{mycolorbox}%
\item%
\textbf{Summary:}%
\newline%
Hey I just had my exam and here are the questions:\newline%
 {-}diVincenzo\newline%
 {-}why do we need anharmonicity\newline%
 {-}condition for a pulse (tau >> 1/(2alpha))\newline%
 {-}initialization\newline%
 {-}measurement line (in fact he only wanted to know about JPA and HEMT no RT components)\newline%
 {-} tomography: how do you do it, physical results, Log Likelihood function%
\item%
\begin{mycolorbox}%
\textbf{Summary:}%
\newline%
In my case, the questions are related to 3 parts: Initialization, Measurement of qubit. Decay mechanism, Surface code.\newline%
 Start: DiVincenzo Criteria (then he picked one of them to ask some more detailed questions)\newline%
 How do we initialize qubit? How long does it take usually? Do we have some other faster ways to achieve that?\newline%
 Draw the readout circuit and describe each component\newline%
 What are the expressions Of kappa(decay rate from LC resonator into the transmission line) and g(coupling strength between qubit and LC circuit)\newline%
 The expression Of the Purcell decay rate\newline%
 Could you briefly describe the measurement induced so glad that I read that problem set in detail yesterday night) In face he wants to\newline%
 see the partial trace after measurement.\newline%
 How to measure s (plot the Quadrature vs In Phase)? What we can quantify by rneasuring s?\newline%
 Tell me some sources of the decay mechanism(TI: coupling to EM environment. coupling to the TLS in defect. quasi particle transition)\newline%
 For the TLS in defect, how can we relate it to the quality factor Q?\newline%
 As for the participation ratio p\_i, how can we decrease it during design and fabrication. (Simply by making the chip bigger)\newline%
 About surface code: draw the circuit diagram to show how to do the parity measurement (2 data qubits + 1 ancillary qubit + some rotation gates + C{-}\newline%
 phase gate)\newline%
 The expression of the logical error probability as the function of the physical error p and surface code distance d?%
\end{mycolorbox}%
\item%
\textbf{Summary:}%
\newline%
DiVincenzo\newline%
 Universal set of gates\newline%
 Why do we need the T{-}gate?\newline%
 Formula for %
$\gamma$%
\newline%
 How to implement single qubit gates\newline%
 Need for attenuation for qubit control pulses\newline%
 Dependence of qubit frequency on magnetic flux\newline%
 %
$T_I$ %
 and %
$T_2$ %
 times, meaning and causes\newline%
 Dispersive read{-}out coupling term in Hamiltonian\newline%
 Why photon shot noise influences %
$T_2$ %
 (photons in read{-}out resonator follow Poissonian statistics)\newline%
 Formula for %
$\xi$%
\newline%
 Formula for %
$Q$ %
 (with participation ratio), the defects that are most important and how to fix these\newline%
 Shor{-}9 code\newline%
 Circuit for bit{-}flip correction (3 data + 2 ancilla)%
\item%
\begin{mycolorbox}%
\textbf{Summary:}%
\newline%
Just had my exam, questions were:\newline%
 {-}DiVincenzo\newline%
 {-}Universal set of gates, why do we need T gate?\newline%
 {-}Draw circuit with qubit \& control line\newline%
 {-}Formula for kappa, what order Of magnitude is it?\newline%
 {-}How do we perform single qubit gates and CNOT?\newline%
 {-}What do we have to do with signal before it reaches chip (attenuation)?\newline%
 {-}Draw levels of attenuation\newline%
 {-}Why do we distribute attenuation over different temperature stages?\newline%
 {-}Draw readout elements (resonator + readout line) in circuit\newline%
 {-}Write down coupling Hamiltonian needed for readout, in what regime is it? (dispersive limit)\newline%
 {-}How does this (referring to coupling Hamiltonian) allow us to perform readout?\newline%
 {-}How would we derive scattering parameters (write down equation of motion)?\newline%
 {-}Detection efficiency, how can we find it out? (measurement induced dephasing)\newline%
 {-}Explain how measurement induced dephasing works and how we get detection efficiency\newline%
 {-}How do we get s? (ramsey experiment)\newline%
 {-}Name reasons for decoherence (Tl and T 2)\newline%
 {-}Explain how quasiparticle tunnelling works\newline%
 {-}Write down formula for Q, explain the terms, how could you improve Q?%
\end{mycolorbox}%
\end{enumerate}%
\newpage%
\begin{center}%
\begin{large}%
\textbf{Fall 2019, Examiner: Christopher Eichler}%
\end{large}%
\end{center}%
\begin{enumerate}%
\item%
\begin{mycolorbox}%
\textbf{Summary:}%
\newline%
Hello guys, I just finished the oral exam. Topics included the five criteria, universal set of gates, how to implement a CZ gate. SQUID and how frequency changes with flux. Then he showed a picture of transmon and asked what are different parts. Finally we discussed sources of noise, particularly material defects. Hope that is helpful!%
\end{mycolorbox}%
\item%
\textbf{Summary:}%
\newline%
Hey,guys! Here's how my exam went.\newline%
Started with 5 requirements and choose two of them.\newline%
For me: coherence time and universal gate.\newline%
What will influence coherence time? (He will go into details depending on what you answer, for example, in the effective two level system, the formula of Q factor and the meaning of each variables )\newline%
what is minimum set of gate to achieve arbitrary operation? \newline%
How to realize CNOT gate? \newline%
Then to explain some figures in the slides (he printed out)\newline%
Benchmarking, exponential decay of the g population\newline%
Surface code (logical operator for 5*5 lattice) \& threshold prob.\newline%
Integration of quadrature, where does the noise come from? How to decrease the noise? Draw the Detection chain\newline%
(And in my exam, he doesn't cover Chapter2{-}Quantized electrical circuits, Chapter3{-}SC qubit and Chapter7{-}QO related experiments, I was not asked about any Hamiltonian and any math derivation.)%
\item%
\begin{mycolorbox}%
\textbf{Summary:}%
\newline%
Hey guys I had the exam. The main question were:\newline%
Di Vincenzo's criteria. How to initialize a qubit. The dephasing source. Ramsey experiment. Dephasing time and experiment. How to calculate the Hamiltonian associated to a pair Of qubit capacitively coupled. Two qubit gates, cz gate. Implementation with the bus resonator%
\end{mycolorbox}%
\item%
\textbf{Summary:}%
\newline%
During my exam we talked about:\newline%
 {-} DiVencenzo•s 5 criteria (obviously)\newline%
 {-} How to perform two qubit operations on two distant qubit (i.e. with other qubits in\newline%
 {-} SWAP gates and their decomposition\newline%
 {-} What a universal set of gates means and of what gates it consists\newline%
 {-} How you can physically apply single qubit gates\newline%
 {-} What limits the width of such a pulse (in order to establish a single qubit gate for example)\newline%
 {-} Speak about the different components of an X{-}mon from a picture\newline%
 {-} Talk about the different components of a picture of the circuit board\newline%
 {-} How to calculate scattering factors\newline%
 {-} Write down equation of motion for this%
\item%
\begin{mycolorbox}%
\textbf{Summary:}%
\newline%
Hi everyone, I had my QSSC exam this morning. My topics cover:\newline%
 1. DiVincenzo's criteria\newline%
 2. Factors influencing qubit coherence time (both relaxation time and decoherence time)\newline%
 3. Two level systems induced relaxation\newline%
 4. Realization of single qubit gates\newline%
 5. Set up for flux control line (attenuation)\newline%
 6. Noise level at room temperature and at the input port of chip (Problem Set 07)\newline%
 7. Calculation of coupling rate \textbackslash{}kappa through charge Inie\newline%
 8. Generaton of Bell state\newline%
 9. How to perform state tomography\newline%
 10. State tomography of Bell state%
\end{mycolorbox}%
\end{enumerate}%
\newpage%
\end{document}