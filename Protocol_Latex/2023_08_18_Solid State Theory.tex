\documentclass{article}%
\usepackage[T1]{fontenc}%
\usepackage[utf8]{inputenc}%
\usepackage{lmodern}%
\usepackage{textcomp}%
\usepackage{lastpage}%
%
\usepackage[svgnames]{xcolor}%
\usepackage{enumitem}%
\usepackage{ragged2e}%
\usepackage[breakable]{tcolorbox}%
\usepackage{graphicx}%
\usepackage[left=1in, right=1in, top=1in, bottom=1in]{geometry}%
\usepackage{eso-pic}%
\usepackage{tikz}%
\newcommand\Watermark{\put(0, 0.5\paperheight){\parbox[c][\paperheight]{\paperwidth}{\centering \rotatebox[origin=c]{45}{\tikz\node[opacity=0.1]{\includegraphics[width=0.7\textwidth]{2023_04_QEC.png}};}}}}%
\newtcolorbox{mycolorbox}{breakable, colback=Gainsboro, coltext=black, opacityfill=0.15, boxsep=0pt, arc=0pt, boxrule=0pt, left=0pt, right=0pt, top=2pt, bottom=2pt, nobeforeafter, fontupper=\normalsize, before upper={\begin{justify}\parindent0pt}, after upper={\end{justify}},}%
\AddToShipoutPictureBG{\Watermark}%
%
\begin{document}%
\normalsize%
\begin{center}%
\begin{Large}%
\textbf{Solid State Theory}%
\end{Large}%
\linebreak%
\end{center}%
\begin{center}%
\begin{large}%
\textbf{Spring 2023, Examiner: Prof. Sigrist}%
\end{large}%
\end{center}%
\begin{enumerate}%
\item%
\begin{mycolorbox}%
\textbf{Summary:}%
\newline%
{-} Stability of metals using the Hartree{-}Fock approach\newline%
{-} Stability of semiconductors\newline%
{-} Why are metals shiny? Optical properties of metals in various regimes\newline%
{-} Relation between the conductivity and the dielectric constant\newline%
{-} The role of interband transitions for optical properties of metals\newline%
{-} Completely transparent metal%
\newline%
\newline%
\textbf{Exam atmosphere:}%
\newline%
The atmosphere is very chill. Sigrist first does some small talk to relax the atmosphere and make you feel comfortable. The exam feels like a conversation. The prof is more interested to see that you understood the physics of the topics, rather than just the math. However remembering formulas helps a lot.%
\end{mycolorbox}%
\end{enumerate}%
\newpage%
\end{document}