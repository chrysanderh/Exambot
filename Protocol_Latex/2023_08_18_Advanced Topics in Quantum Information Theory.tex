\documentclass{article}%
\usepackage[T1]{fontenc}%
\usepackage[utf8]{inputenc}%
\usepackage{lmodern}%
\usepackage{textcomp}%
\usepackage{lastpage}%
%
\usepackage[svgnames]{xcolor}%
\usepackage{enumitem}%
\usepackage{ragged2e}%
\usepackage[breakable]{tcolorbox}%
\usepackage{graphicx}%
\usepackage[left=1in, right=1in, top=1in, bottom=1in]{geometry}%
\usepackage{eso-pic}%
\usepackage{tikz}%
\newcommand\Watermark{\put(0, 0.5\paperheight){\parbox[c][\paperheight]{\paperwidth}{\centering \rotatebox[origin=c]{45}{\tikz\node[opacity=0.1]{\includegraphics[width=0.7\textwidth]{2023_04_QEC.png}};}}}}%
\newtcolorbox{mycolorbox}{breakable, colback=Gainsboro, coltext=black, opacityfill=0.15, boxsep=0pt, arc=0pt, boxrule=0pt, left=0pt, right=0pt, top=2pt, bottom=2pt, nobeforeafter, fontupper=\normalsize, before upper={\begin{justify}\parindent0pt}, after upper={\end{justify}},}%
\AddToShipoutPictureBG{\Watermark}%
%
\begin{document}%
\normalsize%
\begin{center}%
\begin{Large}%
\textbf{Advanced Topics in Quantum Information Theory}%
\end{Large}%
\linebreak%
\end{center}%
\begin{center}%
\begin{large}%
\textbf{Spring 2021, Examiner: L. Del Rio, R. Silva}%
\end{large}%
\end{center}%
\begin{enumerate}%
\item%
\begin{mycolorbox}%
\textbf{Summary:}%
\newline%
You arrive at HCl F2 —IO minutes before the exam (or zoom if that's your jam). There's a paper with some instructions and some paper for you to write on.\newline%
Nuriya will come out the room IO minutes before to give you the topic. She does it with a random number generator so it's truly random and not some topic they haven't heard.\newline%
I got topic 15 (Von Neumann/weak measurements, weak values). This took (l guess) around 7 minutes indeed and then they started asking about other topics.\newline%
Ralph took over the questions from Lidia and started to ask about the resource theory Of noisy operations. Thereafter came clocks: difference between a ticking clock and stopwatch. What the ideal clock is and last question: how do you use this to implement a unitary conditioned on timing%
\end{mycolorbox}%
\item%
\textbf{Summary:}%
\newline%
Got Hardy's experiment as a starter, during the presentation I was never interrupted, until I had gone through what I wanted to say. Lidia then asked about how else we could describe the paradox, wanting to hear weak values but not going into detail.\newline%
Then Ralph asked me how I could erase a qubit that is in a general state, (using a bath), and how to this if I wanted to keep track of energy as well during cooling. He then asked for the general bound on the work cost of erasure, and we went a bit into where it comes from. where I then did the derivation of the inequality in the Szilard's engine scenario. \newline%
In general: the first part was very nice, where they only asked fairly straightforward questions towards the end of the presentation. In the second part, the questions were a bit more specific, and so sometimes they had me swimming as they were jumping between {-}topics" in which I had ordered everything more frequently, and asked for more detailed stuff, (e.g. why is the term in the derivation Of Landauer erasure O, etc).\newline%
They were all very nice, with Lidia being more direct, and Ralph trying everything possible to make sure I understood the question.%
\item%
\begin{mycolorbox}%
\textbf{Summary:}%
\newline%
I got Maxwell's Demon. Like the others said, they let you present the topic without much interruption, I just got one question at the end which was why is the memory in the zero state at the start. I was then asked about Hardy's experiment (Setup, simplified version). Was casual. they were very friendly.%
\end{mycolorbox}%
\item%
\textbf{Summary:}%
\newline%
(+10 min delay). The topic I was allowed to look at beforehand was PPS paradoxes: I presented the setup for a PPS experiment, then defined the ABL rule and went through the 3 well{-}behaved conditions. Then I showed on the example of Hardys paradox how a PPS paradox looks like.\newline%
Then Lidia asked me about what this has to do with contextuality and I stated the Theorem 'logical PPS implies violation of non{-}contextuality'.\newline%
Ralph took over and asked me about Passivity: I defined ergotropy, passivity and the characterization with block diagonal in energy and decreasing eigenvalues. He then asked me to argue why this is true and I presented a sketch of the proof. Then Ralph kindly asked me to define complete passivity and sketch the proof of why the the completely passive states are exactly the Gibbs states. That was a bit a pain in the ass and after him pointing out a few typos I got the right idea. At the end, Lidia asked me to explain how we can implement a unitary using an ideal clock. There I presented the calculation from the lecture. I was again asked to derive the result.\newline%
All in all they were nice but they asked for some quite involved derivationsAA But only from the lecture, non from the exercises.%
\item%
\begin{mycolorbox}%
\textbf{Summary:}%
\newline%
I was given topic 12: Quasi{-}ideal clocks. Started talking about the Peres clock and how it acts in time what the issues with it are and what the limits for d inf and omega'd = const. are. Then I moved on to the improvement done by Ralph etc, the Gaussian wavepacket clock (warning: I was going to mention that Ralph came up with it but much to my embarassment instead did an oopsie and called it the Salecker Wigner clock upon which I got stopped by Ralph who cleared it up {-} I hope I did not offend him)\newline%
He then asked how we do the time evolution, what approximations we have to do (Poissonian formula, although I was not able to write it down, but I did know it's an extension to infinite dimensional spaces), why we need it (so that the var {-} l/var). This last part was a quite a mess...\newline%
Then we moved on to Thermal Operations and I hope I said all the relevant stuff. And last I got to talk about the Pigeonhole paradox very quickly and then the time was over.%
\end{mycolorbox}%
\end{enumerate}%
\newpage%
\end{document}