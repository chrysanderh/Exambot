\documentclass{article}%
\usepackage[T1]{fontenc}%
\usepackage[utf8]{inputenc}%
\usepackage{lmodern}%
\usepackage{textcomp}%
\usepackage{lastpage}%
%
\usepackage[svgnames]{xcolor}%
\usepackage{enumitem}%
\usepackage{ragged2e}%
\usepackage[breakable]{tcolorbox}%
\usepackage{graphicx}%
\usepackage[left=1in, right=1in, top=1in, bottom=1in]{geometry}%
\usepackage{eso-pic}%
\usepackage{tikz}%
\newcommand\Watermark{\put(0, 0.5\paperheight){\parbox[c][\paperheight]{\paperwidth}{\centering \rotatebox[origin=c]{45}{\tikz\node[opacity=0.1]{\includegraphics[width=0.7\textwidth]{2023_04_QEC.png}};}}}}%
\newtcolorbox{mycolorbox}{breakable, colback=Gainsboro, coltext=black, opacityfill=0.15, boxsep=0pt, arc=0pt, boxrule=0pt, left=0pt, right=0pt, top=2pt, bottom=2pt, nobeforeafter, fontupper=\normalsize, before upper={\begin{justify}\parindent0pt}, after upper={\end{justify}},}%
\AddToShipoutPictureBG{\Watermark}%
%
\begin{document}%
\normalsize%
\begin{center}%
\begin{Large}%
\textbf{Statistical Physics}%
\end{Large}%
\linebreak%
\end{center}%
\begin{center}%
\begin{large}%
\textbf{Fall 2020, Examiner: Gianni Blatter}%
\end{large}%
\end{center}%
\begin{enumerate}%
\item%
\begin{mycolorbox}%
\textbf{Summary:}%
\newline%
Started with Bosons and Fermions\newline%
 Prof: What can you tell me about bosons and fermions?\newline%
 B: Best to describe them in the grand canonical ensemble\newline%
 I wrote down formulas for the grand partition function, pressure, number, occupancy\newline%
 Blatter wanted to plot the occupancy as a function of energy, T=0, T>0\newline%
 Then he wanted a formula for the fermi energy (eq 6.18)\newline%
 Then he wanted the formula for chemical potential for fermions\newline%
 Then he wanted the behaviour of chemical potential for bosons, how it it diverges and how the plot looks like\newline%
 Then on to massive Bosons\newline%
 G: What can you tell me about massive?\newline%
 B: Well what do you want to know?\newline%
 G: You have to know something about massive bosons\newline%
 B: Well okay, let‘s start with Ideal Bose gas\newline%
 Wrote again the pressure, and density\newline%
 Talked about how if n*lambda\^{}3 is bigger than 2.612 then we get a macroscopic value for the ground state occupancy\newline%
 G: show me the phase diagram!\newline%
 B: Draw the p{-}t diagram and talked about how we can condense, the forbidden region and how the behaviour is\newline%
 On to Magnetic Systems\newline%
 G: What does a phase diagram look for a magnetic system\newline%
 B: Draw a h{-}T diagram showing the first and second order phase transitions\newline%
 G: how does the order parameter behave as a function of temperature?\newline%
 B: wrote down the formula and talked about the value for Beta at Critical and tricritical\newline%
 G: How does the first order jump behave when close to T\_C\newline%
 B: paused for a few seconds, now knowing what he was referring to.\newline%
 G: Draws some lines on the diagram\newline%
 B: Ahh, as a function of h 1/delta where in MFT delta=3\newline%
 On to scaling relations\newline%
 G: How are the scaling exponents related\newline%
 B: They are related through these four scaling relations, wrote down all the relations (chap 13.1)\newline%
 G: Widom had an ansatz about the behaviour around the critical point (data collapse). What is this ansatz?\newline%
 B: wrote down the formula explaining each term\newline%
 G. how is the gap exponent related to our old exponents?\newline%
 B. Gave formula for both that (13.31 ) and relation to lambda exponents\newline%
 Very briefly about RG\newline%
 G: Aa good remark. How do we get these remaining exponents?\newline%
 B: RG, wrote down the RG eqs. And talked about the linearization and the eigenvalue problem to get the remaining exponents\newline%
 G: Very well, time’s up!%
\end{mycolorbox}%
\item%
\textbf{Summary:}%
\newline%
Start:\newline%
 • Explain the 3 laws Of TD\newline%
 • Ideal gas equation of state\newline%
 • How to calculate entropy of the ideal gas\newline%
 Calculate the free energy Of the ideal gas (hamiltonian {-}4 partition function (canonical) —i free energy • )\newline%
 • What will happen when the gas parameter is very large?\newline%
 Go to the Quantum case\newline%
 • partition function for fermions and bosons\newline%
 • Specific heat for both cases (to check that the 3rd law is fulfilled; for bosons he wants TA(d/n))\newline%
 • Draw the Cv as a function of T for BEC;\newline%
 Go to BEC\newline%
 • How to calculate T\_BE?\newline%
 Order parameter?\newline%
 • Similar transition in magnetic system? (Heisenberg; BEC cannot happen below 30, he also asked why )\newline%
 GO to magnetic PT\newline%
 • Hamiltonian for XY model\newline%
 • What is the interesting transition in 2D and describe it (BKT transition);\newline%
 What is QLRO; Write down the formula for it (Algebraical decay correlator, here he really wanted me to get the full formula, i.e. (8.75))\newline%
 • What's the anomolous exponent for BKT transition? (Neta{-}1/4)\newline%
 Go to exponents and scaling law\newline%
 Write down the scaling laws and the exponents for the simplest model (MFT)\newline%
 Do MFT exponents fulfill these laws? (yes except the Josephson)\newline%
 When do they fulfill Josephson?\newline%
 What about tricritical point?%
\item%
\begin{mycolorbox}%
\textbf{Summary:}%
\newline%
These were the topics of my exam:\newline%
 Bosons, Fermions\newline%
 • Partition function, eq. of state, gas parameter\newline%
 • Dilute case: eq. for p for Fermion gas and Boson gas, statistical interaction\newline%
 • Dense case for Fermions: eq. for p, U, Cv\newline%
 BEC\newline%
 • Phase diagrams, ways to cool down in the graph\newline%
 • Equivalent Phase diagram for Van der Waals gas\newline%
 • How to determine T\_BE\newline%
 • What is the order parameter + Graph\newline%
 MFT\newline%
 draw full m{-}h{-}t diagram\newline%
 • What happens at Tc?\newline%
 • critical exponents, derived form landau functional\newline%
 Scaling\newline%
 • Widom scaling: data collapse, Ansatz, Widom law%
\end{mycolorbox}%
\end{enumerate}%
\newpage%
\end{document}