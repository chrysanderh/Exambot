\documentclass{article}%
\usepackage[svgnames]{xcolor}
\usepackage{enumitem}
\usepackage{ragged2e}
\usepackage[T1]{fontenc}%
\usepackage[utf8]{inputenc}%
\usepackage{lmodern}%
\usepackage{textcomp}%
\usepackage{lastpage}%
%
\usepackage{graphicx}%
\usepackage[left=1in, right=1in, top=1in, bottom=1in]{geometry}%
\usepackage{eso-pic}%
\usepackage{tikz}%

\newcommand\Watermark{
	\put(0, 0.5\paperheight){
		\parbox[c][\paperheight]{\paperwidth}{
			\centering \rotatebox[origin=c]{45}{
				\tikz\node[opacity=0.2]{\includegraphics[width=0.7\textwidth]{2023_04_QEC.png}};
			}
		}
	}

}%
\AddToShipoutPictureBG{\Watermark}%

\usepackage{tcolorbox}
\newtcolorbox{mycolorbox}{
    colback=Gainsboro,            % Change 'blue' to the desired color for the box background
    coltext=black,           % Text color (black in this case)
    opacityfill=0.25,         % Change the opacity value (0 to 1) for the colorbox
    boxsep=0pt,              % Space between text and colorbox
    arc=0pt,                 % Round corners (0pt for straight corners)
    boxrule=0pt,             % Border width (0pt for no border)
    left=0pt, right=0pt,     % Distance from text to the left and right edges of the colorbox
    top=2pt, bottom=2pt,     % Distance from text to the top and bottom edges of the colorbox
    nobeforeafter,           % Don't add extra space before or after the box
    fontupper=\normalsize,    % Set the font size for the box
    before upper={\begin{justify}\parindent0pt},
    after upper={\end{justify}},
}

%
\begin{document}%
\normalsize%
\begin{enumerate}%
\item%
He does not let you choose a topic%
\item%
I had the exam this morning, here is how it went:%
\item%
He started by asking how we arrive at the Jaynes{-}Cummings Hamiltonian from the minimal coupling Hamiltonian. He asked about the LWA, then we moved on to the eigenstates of JC. He then asked me about the Purcell regime (what are the relevant parameters), how the spectrum of JC changes in the Purcell regime (no clue), and how it affects the correlation function. He kept asking questions related to the Purcell regime but I had no idea so he changed the topic. %
\item%
We moved on to ion trapping, why in that case the dipole approximation is still valid and what are the relevant physical quantities. %
\item%
He also asked me a few questions about Rabi oscillations and collapse and revival.%
\item%
I had very similar topics to this%
\item%
I would say it very important that you know all assumptions in each regime very well and especially know how to explain them%
\item%
I would also say he does not give you a lot of hints when you are stuck so if you are stuck try saying things that you might think are related because then he will tell you if they are%
\item%
At least that was my experience%
\item%
\begin{mycolorbox}
He asked me to talk about JC and Tavis{-}Cummings, and the differences between them. He wanted to know how we can differentiate whether there is one or many atoms in the cavity by analyzing the leakage (g2 is apparently not the answer he was looking for, I got stuck there for 15min and he didn't wanna move on for some reason, then he revealed the solution but I didn't understand it ). Then he asked about the Purcell regime, he especially wanted to hear the term "large cooperativity". In the end he wanted to know how an two atoms in the dark superpos state can decay and how this depends on the distance between the atoms. I had no clue, then he gave the "hint" that I should consider the wavelength of the e<{-}>g transition which didn't help. He then revealed the…%
\end{mycolorbox}
\item%
I also had a similar experience.. was asked about JC and it’s approximations in detail. What is important in JC and why do we consider it. I said about Rabi oscillations … and then went on to collapse and revival. He wanted me to tell him what happens if we increase the photons in the cavity and how it affects the revival and collapse time. %
\item%
Also talked a bit about Tavis Cummings. %
\item%
And then at the end about dressed states, why we have antibunching and why we don’t have it when we only have a cavity (without emitter). %
\item%
He also asked many times why are these things important for someone who is not in the quantum optics domain … I am not sure what he expected for an answer …%
\end{enumerate}%
\end{document}